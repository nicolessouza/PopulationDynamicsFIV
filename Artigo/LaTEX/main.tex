
\documentclass{article}
%\documentclass[article]{abntex2}
\usepackage[alf]{abntex2cite}
\usepackage[brazil]{babel}
\usepackage[left=2.5cm,top=2.5cm,right=2.5cm,bottom=2.5cm]{geometry}
\usepackage[utf8]{inputenc}
\usepackage{cancel}
\usepackage{authblk}
\usepackage{graphicx, color}
\usepackage{hyperref}
\usepackage{glossaries}
\usepackage{xifthen} 
\usepackage{tikz}
\usetikzlibrary{arrows}
\usepackage{tikz, tkz-euclide,  pgfplots}
\usepackage{tikzscale}
\usepackage[square,sort,comma,numbers]{natbib}
%[square,sort,comma,numbers]
\usepackage{url}

\makeglossaries
\newacronym{rna}{RNA}{ácido ribunocleico}
\newacronym{dna}{DNA}{ácido desoxirribunocleico}
\newacronym{fiv}{FIV}{vírus da imunodeficiência felina}
\newacronym{hiv}{HIV}{vírus da imunodeficiência humana}
\newacronym{pgl}{PGL}{glicolipídeo fenólico}
\newacronym{aids}{AIDS}{síndrome da imunodeficiência adquirida}
\newacronym{arc}{ARC}{complexo relacionado à AIDS}
\newacronym{si}{SI}{sistema de modelagem suscetível-infectado}


\title{Dinâmica Populacional do Vírus da Imunodeficiência Felina em
Populações de Gatos Domésticos}
\author[1]{Daniel Jacob Tonn}
\author[1]{Nicole dos Santos de Souza}
\affil[1]{Escola de Matemática Aplicada - FGV EMAp, Rio de Janeiro, RJ}
\date{Setembro de 2022}

\begin{document}
\maketitle
\begin{center}
    \textbf{RESUMO}
\end{center}


\noindent Este artigo se dedica à compreensão da circulação do Vírus da Imunodeficiência Felina (FIV), um
retrovírus felino homólogo ao Vírus da Imunodeficiência Humana (HIV), em populações de gatos domésticos. Usando equações diferenciais de primeira ordem com base no modelo SI (Suscetível-Infectado), a dinâmica populacional que será apresentada foi abordada sob duas perspectivas: com e sem transmissão vertical. Como a doença dos felinos não tem cura - e pode comprometer a saúde pública, mesmo que de forma indireta - um modelo epidemiológico que permita entender melhor a sua propagação é de extrema utilidade.
\\

\noindent
\text{PALAVRAS-CHAVE: Felinos; Modelo epidemiológico; Retrovírus; Saúde pública; Transmissão.}

%%%%%%%%%%%%%%%%%%%%%%%%%%%%%%%%%%%%%%%%%% GLOSSÁRIOS %%%%%%%%%%%%%%%%%%%%%%%%%%%%%%%%%%
\newpage
% \printglossary[type=\acronymtype]
\printglossary[title ={Lista de Abreviaturas e siglas}]
\newpage

\tableofcontents

\newpage
\section*{INTRODUÇÃO}
\addcontentsline{toc}{section}{INTRODUÇÃO}
Os estudos de circulação da família de vírus  \textit{retroviridae}, popularmente chamados de retrovírus, possuem desfalque quando comparados a trabalhos sobre outros parasitas. De fato, modelar o sistema de propagação de um retrovírus em uma população de mamíferos não é uma tarefa trivial. O período de soropositividade muito longo, bem como a sucessão de diferentes estágios clínicos induzem a necessidade de modelos muito específicos \cite{base}. Além disso, o funcionamento das populações hospedeiras pode ser demasiadamente particular, fazendo com que sua modelagem demande uma boa compreensão dos padrões espaciais e sociais dos hospedeiros.\\

\noindent Neste artigo a questão de interesse é a propagação do \gls{fiv} em gatos domésticos (\textit{Felis Catus}). A doença causada por esse retrovírus compromete o sistema imune do gato de forma parecida com o que o \gls{hiv} faz com seres humanos, por isso é considerado o equivalente da \gls{aids}  para os felinos. Transmitida de forma horizontal, geralmente através de mordidas e arranhões oriundos de  disputas territoriais e de fêmeas da espécie,  a \gls{fiv} possui uma frequência maior de casos em machos não castrados com acesso às ruas.\\

\noindent Por se tratar de um lentivírus e ser específico para a espécie, o \gls{fiv} não representa riscos à saúde pública de forma direta . Sabe-se, no entanto, que o animal que contrai a infecção se torna imunossuprimido, correndo o risco de adquirir doenças secundárias como a toxoplasmose e a criptosporidiose ativa que são transmissíveis para humanos. Dessa forma, revela-se uma ameaça para o bem-estar comum - não só dos felinos, mas dos seus tutores\cite{rayane}. \\

\noindent Ademais, uma vez que a imunodeficiência causada pelo retrovírus nos felinos não tem cura, um modelo epidemiológico que permita compreender melhor a propagação da doença - com objetivo de traçar estratégias com relação ao
controle e prevenção da infecção - é indispensável.
Porém, alguns desafios serão encontrados para tal estudo. A população hospedeira que será trabalhada apresenta um alto grau de variabilidade com relação à distribuição geográfica e condições de sobrevivência, bem como questões sociais ligadas aos seus tutores que precisam ser levadas em consideração. Em vista disso, muitos aspectos significativos devem ser omitidos do estudo. Apesar das deficiências, é procurado um modelo que seja capaz de gerar respostas relevantes quanto ao impacto (em termos de redução do número de indivíduos hospedeiros devido à doença) do lentivírus nas populações de gatos, e sobre a influência do funcionamento da população de felinos na propagação do vírus.  \\

\newpage
\section{Revisão Bibliográfica}
\subsection{Hospedeiros e transmissão} 
\subsubsection{Sobre a distribuição geográfica dos hospedeiros}
Com 101,1 milhões de animais domésticos o Brasil fica em segundo lugar no ranking de maiores criadores de animais domésticos ao redor do mudo, estando atrás apenas dos Estados Unidos que totalizam 146 milhões. A população de gatos brasileira ocupa cerca de 17,7\%  das residências brasileiras, somando 22 milhões de indivíduos com os números crescendo cada vez mais em virtude da independência, higiene e facilidade de criação \cite{gomes}.\\

\noindent A distribuição geográfica do \gls{fiv} é de caráter mundial, com concentrações em ambientes de maior índice de felinos com comportamento agressivo e menos frequente em regiões de baixa densidade populacional ou localidades em que os gatos não possuem contato com os felinos de rua. A distribuição das fêmeas se dá especialmente de acordo com a distribuição humana em virtude de fontes alimentícias e para criação dos filhotes, enquanto a distribuição dos machos (pelo menos no período reprodutivo) se dá de acordo com a distribuição das fêmeas receptivas. Os gatos de rua se concentram em grandes áreas onde há fornecimento de algum recurso alimentar \cite{base}. Assim, o padrão de habitação humana é o
principal fator que determina os padrões sociais e
reprodutivos das populações de gatos domésticos.\cite{base}\\

\subsubsection{Sobre a transmissão da doença}
\noindent A doença causada pelo \gls{fiv} é imunossupressora e  transmitida horizontalmente através de mordidas e arranhões com a troca de sangue e/ou saliva passada de indivíduos infectados para suscetíveis, sendo que durante a amamentação pode ocorrer a transferência de anticorpos para os filhotes \cite{rayane}. Proporcional ao estágio da doença na mãe, os filhotes podem estar mais ou menos suscetíveis à contaminação. O comportamento agressivo é oriundo normalmente de disputas por alimento, território ou fêmeas da espécie, resultando em uma frequência maior de casos em machos não castrados com acesso às ruas.\\

\noindent O vírus da imunodeficiência felina pertence à família dos \textit{retroviridae}. Na mesma família encontramos o vírus da imunodeficiência humana (\gls{hiv}), que em contraposição ao \gls{fiv}, pode ser transmitido através de relações sexuais.
Além de fornecer um modelo relevante para estudos de \gls{hiv}   , o \gls{fiv} é de particular interesse na ciência veterinária, pois sua infecção dá origem a uma ampla gama de sinais clínicos.
Foi feita uma tentativa de delinear o curso clínico da infecção pelo \gls{fiv} em uma série de cinco estágios análogos aos da infecção pelo \gls{hiv} em humanos
\cite{base}. Primeiro, o gato infectado sofre de um estágio agudo que ocorre várias semanas após a infecção e dura de 4 a 16 semanas. Apesar de se recuperar do estágio primário da doença, praticamente todos os gatos infectados com \gls{fiv} se tornam portadores do vírus ao longo da vida. A primeira fase é seguida por uma fase de portador assintomática que dura meses a anos, durante a qual o comportamento do gato não parece ser afetado. Segue-se linfadenopatia generalizada persistente (\gls{pgl}), complexo relacionado à \gls{aids} (\gls{arc}) e a própria \gls{aids}, caracterizada por distúrbios diversos e infecções oportunistas. Espera-se que a maioria dos gatos infectados morra por "causas naturais" (acidentes rodoviários, caça, envenenamento, etc.) antes de atingir os estágios terminais da infecção por \gls{fiv}. Eles devem ter tempo para transmitir o vírus, mas não para morrer por conta dele. Portanto, esperamos que o vírus tenha um baixo impacto nas populações naturais de gatos domésticos.

\subsection{Trabalhos na área}
Há algumas possibilidades de caminhos a seguir para modelar um problema desse tipo. Em \textit{A diffusive SI model with Allee effect and application to FIV}\cite{difusao}, por exemplo, utiliza-se um \gls{si} de equações diferenciais parciais com a presença do efeito Allee. Em \textit{Dynamics of two feline retroviruses (FIV and FeLV)
within one population of cats} \cite{franck} foi tomada a decisão de modelar duas doenças em conjunto e verificar sua relação. Já o artigo \textit{"Stochastic dynamics of feline immunodeficiency virus"} \cite{estocastico}, ao contrário dos outros, considera transmissão vertical no modelo e utiliza técnicas de dinâmica estocástica para os cálculos.

\section{Metodologia}
\subsection{O modelo}
O modelo que será utilizado aqui foi embasado nos estudos de Anderson e May\cite{anderson}, descrito em \cite{base} e se trata da dinâmica de uma população de gatos domésticos representada por um conjunto de equações diferenciais de primeira ordem. Considere $N$ como  o número total de gatos no tempo $t$ e $K$ a capacidade máxima de indivíduos no habitat. Assume-se a taxa de crescimento linear da população dada por $r=b-m$, onde $b$ é a taxa de natalidade e $m$ a taxa de mortalidade por causas naturais. A taxa de mortalidade é relacionada linearmente com a proporção de indivíduos e segue a forma $
m+\frac{rN}{K}
$. Com a população livre do \gls{fiv}, a taxa de crescimento da população é dada pela equação logística familiar de Verlhust \cite{helmut}:
\begin{equation} \label{propcres}
\frac{dN}{dt} = rN\left( 1-\frac{N}{K} \right)
\end{equation}

\noindent Para introduzir a doença na população, representa-se $S = S(t)$ para os gatos suscetíveis ao \gls{fiv} e $I = I(t)$ para os gatos infectados, onde $N = S + I$. Para evitar maior complexidade nos cálculos, são considerados todos os estágios (agudo, soropositivo, \gls{pgl}, \gls{arc} e \gls{aids}) como um único. De fato, os três últimos períodos são curtos e implicam na comorbidade do felino, fazendo com que ele esteja distante de disputas de natureza agressiva por território ou acasalamento e também de situações que o levem à morte por causas naturais.

\noindent A transmissão do \gls{fiv} depende diretamente da taxa de encontros entre os animais - a qual será representada pela constante ($\rho$) - da taxa de encontros que resultam em mordidas ou arranhões ($\beta$) e da taxa de eficácia de transmissão dessa doença caso exista contato ($c$). É relativamente simples estimar $c$ através de experimentos em laboratório, mas as taxas $\rho$ e $\beta$ são fortemente influenciadas pela distribuição geográfica dos animais e pelo seus comportamentos. Por conta da complexidade que envolve estimar essas variáveis separadamente, considera-se, para fins de simplificação do modelo, uma constante $\Omega$ que equivale a unificação das três taxas propostas anteriormente ($\rho\beta c$) cuja informação é resumida em taxa de mordidas efetivas para a transmissão do vírus. A taxa de mortalidade pelo \gls{fiv} é independente da densidade populacional e dada por $\alpha$ que corresponde ao inverso do período infeccioso ($\frac{1}{\alpha}$).\\

\noindent Para modelar o processo de transmissão, é preciso que ocorra um encontro entre um indivíduo infectado e um indivíduo suscetível. A proporção de infectados na população é $\frac{I}{I+S}$. Sendo assim, o contágio horizontal pode ser representado da forma:
$$\frac{\rho \beta cSI}{I+S}$$
ou simplesmente 
$$\frac{\Omega SI}{I+S}$$

\noindent Há agora dois caminhos que podem ser seguidos. O artigo base para esse modelo desconsidera a transmissão vertical nos cálculos pois ela, teoricamente, não terá grande impacto nos resultados. Porém, biologicamente, essa possibilidade de transmissão existe. Haverá, nesse momento, uma divisão em casos do modelo. Uma das versões irá considerar a transmissão vertical nas equações e a outra irá omitir essa parte. Em ambos os casos, será assumido que não há possibilidade de recuperação (um indivíduo não retorna ao compartimento de suscetíveis uma vez que ingressa em infectados).

\subsubsection{Sem transmissão vertical}
Será tratado aqui o caso em que todos os filhotes nascem suscetíveis (femêas infectadas não transmitem na gestação). Assim, há apenas uma taxa de natalidade considerada = {\it b}. O conjunto de equações diferenciais de primeira ordem
que descreve a dinâmica populacional de gatos domésticos na presença da \gls{fiv} é dada por:

\begin{equation} \label{suscetiveis}
 \frac{dS}{dt}=b(S + I) - mS - \frac{rNS}{K} - \frac{\Omega SI}{N}
\end{equation}

\begin{equation} \label{infectados}
 \frac{dI}{dt}=  \frac{\Omega SI}{N} - mI - \frac{rNI}{K} - \alpha I
\end{equation}

A equação da população total é obtida através da soma da equação~\ref{suscetiveis} com a equação~\ref{infectados}

\begin{equation} \label{poptotal}
\frac{dN}{dt} = rN\left( 1-\frac{N}{K} \right) - \alpha I
\end{equation}

O que corresponde à equação que vimos em ~\ref{propcres} com o acréscimo de uma taxa de mortalidade pela doença.

A representação compartimental pode ser vista na imagem abaixo:

\begin{figure}[!ht]
\centering
%domínio da função
\includegraphics{gráficos/grafo2.tikz}
\end{figure}
 
\subsubsection{Com transmissão vertical}
Fundamentada com teoria, a transmissão vertical existe na situação que está sendo modelada e será considerada nos cálculos neste caso. Ou seja, 
os recém-nascidos dos infectados estarão na classe infecciosa. Há a inserção de uma nova constante no modelo {\it $b_1$} que corresponde à taxa de natalidade de gatos infectados ({\it b} é a taxa de natalidade de gatos suscetíveis). Com isso, a razão $0 \leq \lambda := \frac{b_1}{b} \leq 1$ descreve a capacidade reprodutiva reduzida de
gatos infectados. Nota-se que $\lambda = 0$ indica que os gatos infectados perdem a capacidade de reprodução - neste caso, não há transmissão vertical. Por sua vez, $\lambda = 1$ significa que os infectados não experimentam redução na aptidão reprodutiva ($b = b_1$).\\

\noindent Dessa forma, o conjunto de equações que representa a dinâmica populacional considerando transmissão vertical é:

\begin{equation} \label{suscetiveis2}
 \frac{dS}{dt}=bS - mS - \frac{rNS}{K} - \frac{\Omega SI}{N}
\end{equation}

\begin{equation} \label{infectados2}
 \frac{dI}{dt}=  \frac{\Omega SI}{N} + b_1I - mI - \frac{rNI}{K} - \alpha I
\end{equation}

\noindent A dinâmica da população total é obtida através da soma da equação~\ref{suscetiveis2} e ~\ref{infectados2} que está descrita abaixo:

$$\frac{dS}{dt} + \frac{dI}{dt} = bS - mS - \frac{rNS}{K} - \cancel{\frac{\Omega SI}{N}} + \cancel{\frac{\Omega SI}{N}} + b_1I - mI - \frac{rNI}{K} - \alpha I $$
$$\frac{d(S + I)}{dt} = bS + b_1S -m(I + S) + rN(-\frac{S}{K} -\frac{I}{K}) - \alpha I $$
$$\frac{dN}{dt} = bS + b_1S -mN + rN(-\frac{N}{K}) - \alpha I $$\\

\noindent Nota-se que $bS + b_1I$ corresponde à taxa de natalidade total da população, incluíndo infectados e suscetíveis. Por isso, podemos considerar como $bN$ (pois $b$ está relacionada com $r = b- m$ que é o crescimento da população como um todo) e a equação tomará a forma:
$$ \frac{dN}{dt} = bN - mN + rN(-\frac{N}{K}) - \alpha I $$
$$ \frac{dN}{dt} = rN(1 - \frac{N}{K}) - \alpha I $$

coincidindo com a equação final correspondente ao modelo sem transmissão vertical.\\
\noindent O esquema compartimental para ilustrar melhor a dinâmica pode ser visto abaixo:

\begin{figure}[!ht]
\centering
%domínio da função
\includegraphics{gráficos/grafo.tikz}
\end{figure}
\newpage
% as natalidades são igualmente distribuídas entre mães sadias e mães infectadas, uma afirmação não fundamentada, pois como discutido anteriormente, a doença é mais frequente em machos que disputam por território e fêmeas, não havendo transmissão sexual.\\
%  Segundo pesquisas realizadas por Rogers e Hoover  cerca de 60 $\%$ das crias de uma ninhada foram diagnosticadas com o vírus\cite{arlin}. 
%  Logo, no modelo da equação~\ref{propcres} é modificada a proporção de recém nascidos suscetíveis, passando a ser $b(S+0,4I)$ e adicionada uma taxa de crescimento nos infectados de $0,6bI$. 
%  A modelagem das populações saudável e adoentada podem ser descritas pelas equações abaixo:
%  \begin{equation} \label{suscetiveis}
%  \frac{dX}{dt}=b(S + 0,4I) - mS - \frac{rNS}{K} - \frac{\Omega SI}{N}
%   \end{equation}
%   \begin{equation} \label{infectados}
%  \frac{dY}{dt}= 0,6bI+\frac{\Omega SI}{N} - mY - \frac{rNY}{K} - \alpha Y
%   \end{equation}
   
% A equação da população total é obtida através da soma da equação~\ref{suscetiveis} com a equação~\ref{infectados}
% \begin{equation} \label{poptotal}
% \frac{dN}{dt} = rN\left( 1-\frac{N}{K} \right)
% \end{equation}

\subsubsection{Análise dimensional}
Abaixo, exposta em uma tabela, encontra-se a análise dimensional dos modelos. A unidade de tempo ainda não foi caracterizada e os indivíduos são gatos domésticos a serem especificados por  região ou característica futuramente. As lacunas em que a unidade não está informada contém uma constante do modelo, medida ou estimada.
\begin{table}[!h]
    \centering
\begin{tabular}{| l | r | r |}
\hline
Parâmetro & Unidade & Descrição\\
\hline
S = S(t) & indivíduos & quantidade de indivíduos suscetíveis no tempo t\\
I = I(t) & indivíduos & quantidade de indivíduos infectados no tempo t\\
N & indivíduos & quantidade total de indivíduos no tempo t\\
K & indivíduos & capacidade máxima de indivíduos no habitat\\ 
b         & - & constante referente à taxa de nascimentos de indivíduos suscetíveis\\
$b_1$        & - & constante referente à taxa de nascimentos de indivíduos infectados\\
m        & - & contante referente à taxa de mortes por causas naturais\\
r = (b - m) & & constante referente à taxa de crescimento linear da população\\
$\alpha$ & & constante referente à taxa de mortalidade por FIV \\
$\frac{1}{\alpha}$ & unidades de tempo & comprimento do período infeccioso\\
$\rho$ & - & constante referente à taxa de encontros entre os animais\\
$\beta$ & - & constante referente à taxa de encontros que resultam em mordidas/arranhões\\
c & - & constante referente à taxa de eficácia da transmissão do vírus \\
$\Omega = \rho\beta c)$ & - & contante referente à taxa de mordidas efetivas para transmitir FIV\\
$\lambda = \frac{b_1}{b}$ & - & capacidade reprodutiva reduzida dos gatos infectados\\
\hline
\end{tabular}
\end{table}

\newpage 
%\nocite{*}
%\bibliographystyle{IEEEtran}
\bibliography{bibliografia/ref.bib}
\end{document}
